\documentclass{scrartcl}

\usepackage[utf8]{inputenc}
\usepackage[T1]{fontenc}
\usepackage{ nicefrac}
\usepackage[left=1.5cm,right=1.5cm,top=2.8cm,bottom=2.8cm]{geometry}
%\usepackage[]{mathpazo,palatino}
%\usepackage[widespace]{fourier}

% body:            Palatino 10pt
% section titles:  Palatino Bold
% formulas:        Euler-VM
\usepackage{palatino}
\usepackage{eulervm}
\linespread{1.08}  % Palatino needs more leading

% body:            CM Roman 11pt
% section titles:  CM Sansserif Demibold Condensed
% formulas:        CM Math
%\usepackage{ccfonts}
%\setkomafont{sectioning}{\fontfamily{cmss}\fontseries{sbc}\selectfont}


\usepackage[ngerman]{babel}
\usepackage{graphicx}
\usepackage{xcolor}

\usepackage{cancel}


% http://www.latex-community.org/forum/viewtopic.php?f=46&t=21038

\usepackage[bookmarksopen=true,pdfsubject={Statistik},pdftitle={Statistik},pdfauthor={Uwe Ziegenhagen},linkcolor=blue,citecolor=blue,urlcolor=blue,colorlinks=true,pdfproducer={PDFLaTeX},pdfcreator={LaTeX 2e},pdftex,backref]{hyperref}

\usepackage{amsmath,amstext}
\usepackage[]{attachfile}
\setlength{\parindent}{0pt}
\setlength{\parskip}{0pt}

\def\qs{\text{QS}(a,b)}
\def\sm{\sum\limits_{i=1}^{n}}
\DeclareMathOperator{\cov}{Cov}
\DeclareMathOperator{\var}{Var}
\DeclareMathOperator{\E}{E}

\makeatletter
\@ifundefined{pdfoutput}{}{\DeclareGraphicsRule{*}{mps}{*}{}}
\makeatother

\title{Herleitung der Parameter-Gleichungen für die einfache lineare Regression}
\author{Uwe Ziegenhagen}

\begin{document}

\maketitle

\subsubsection*{Historie}

\begin{description}
\item[v1.0] 16.03.2009, erste Version hochgeladen
\item[v2.0] 02.03.2013, einen Vorzeichenfehler beseitigt, diverse Gleichungen und Erläuterungen zum besseren Verständnis hinzugefügt.
\end{description}

\section{Einführung}

Aus der Wikipedia: Die Regressionsanalyse ist ein statistisches Analyseverfahren mit dem Ziel, Beziehungen zwischen einer abhängigen und einer oder mehreren unabhängigen Variablen festzustellen. 

Allgemein wird eine metrische Variable $Y$ betrachtet, die von einer zweiten Variablen $x$ abhängt. Üblicherweise ist $x=(x_1,...,x_n)^T$ ein $n$-dimensionaler Vektor, wobei die einzelnen $x$-Werte untereinander unabhängig sind. Im eindimensionalen Fall spricht man von einer einfachen linearen Regressionsanalyse, in Dimensionen größer gleich zwei von einer multiplen Regressionsanalyse.

Bei der einfachen linearen Regression liegen Wertepaare der Form $(x_i, y_i)$ mit $ i=1,\ldots, n$ vor. Als Modell wählt man
    
\begin{equation}
	 Y_i = b + a x_i + \epsilon_i
\end{equation}
   
man nimmt somit einen linearen Zusammenhang zwischen $x_i$ und $Y_i$ an. Die Daten $y_i$ werden als Realisierungen der Zufallsvariablen $Y_i$ angesehen, die $x_i$ sind nicht stochastisch, sondern Messstellen. Ziel der Regressionsanalyse ist in diesem Fall die Bestimmung der unbekannten Parameter $a$ und $b$.

Die Vorgehensweise bei der linearen Regression veranschaulicht folgende Grafik. Gegeben sind Wertepaare $x_i,y_i$, als schwarze Punkte eingezeichnet. Grün sind die Werte $(\hat{x},\hat{y})$ die durch die lineare Regressionsfunktion errechnet werden. Die roten Linien symbolisieren die Abweichungen\footnote{Es ist egal, ob man $y_i-\hat{y}_i$ oder $\hat{y}_i-y_i$ schreibt, durch die Quadrierung heben sich eventuelle negative Vorzeichen auf.}  $e_i=y_i-\hat{y}_i$ dieser durch die Gleichung bestimmten Punkte von den wahren Punkten. Aufgabe bei der Bestimmung der Parameter ist es nun, $a$ und $b$ so zu wählen, dass die Summe QS der quadrierten Abweichungen -- also $\sum_{i=1}^n (y_i-\hat{y}_i)^2$ -- minimal wird. 

\begin{center}
\includegraphics[width=0.75\textwidth]{lr.1}
\end{center}

\section{Herleitung der Gleichungen}


\begin{align}\label{eq:1}
\qs	&= \sum_{i=1}^{n} e_i^2 = \sm (y_i - \hat{y}_i)^2\\
								&= \sm \Big(y_i - [ax_i+b]\Big)^2 \label{eq:2}
\end{align}

Da wir die optimalen Werte für die Minimierung dieser Quadratsumme erhalten wollen, bilden wir die partiellen Ableitungen nach $a$ und $b$. Vorher können wir jedoch Gleichung \ref{eq:1} vereinfachen. Mit Hilfe der 2. Binomischen Formel\footnote{\quad 2. Binomische Formel: \((s-t)^2 = s^2 -2st + t^2\)} lösen wir \ref{eq:2} auf:

\begin{equation}
	\qs = \sm \Big(y_i^2 - 2y_i (ax_i+b) + (ax_i +b)^2\Big)
\end{equation}

Da der Term \((ax_i +b)^2\) der 1. Binomischen Formel\footnote{\quad 1. Binomische Formel: \((s+t)^2 = s^2 +2st + t^2\)} entspricht, lösen wir auch diesen auf und vereinfachen:

\begin{equation}
	\qs = \sm \Big(y_i^2 - 2ax_iy_i - 2by_i + a^2x_i^2 + 2abx_i + b^2\Big)\label{eq:simp}
\end{equation}

Ausgehend von Gleichung \ref{eq:simp} bilden wir jetzt die partiellen Ableitungen nach $a$ und $b$:

\begin{align}
\frac{\partial \qs}{\partial a}	&= \sm (-2x_iy_i + 2ax_i^2 + 2bx_i) \label{eq:parta1}\\
								&= 2 \sm x_i(-y_i + ax_i + b) \label{eq:parta2} 
\end{align}

\begin{align}
\frac{\partial \qs}{\partial b}	&= \sm (-2y_i+2ax_i +2b) \label{eq:partb1}\\
								&= 2 \sm (ax_i + b -y_i) \label{eq:partb2}
\end{align}

Wenn wir Gleichung \ref{eq:partb2} nullsetzen und auflösen, erhalten wir 

\begin{alignat}{3}
2\sm ax_i &+ 2\sm b &&- 2 \sm y_i \quad &&= 0 \\
2\sm ax_i &+ \enskip 2nb &&- 2 \sm y_i \quad &&= 0 \\ 
2nb &= 2 \sm y_i &&- 2\sm ax_i && 
\end{alignat}
%& 2 \sm y_i &&- 2\sm ax_i &&= 2nb

Auflösen nach $b$ (durch $2n$ teilen) gibt (zusammen mit der Tatsache, dass das arithmetische Mittel allgemein als $\bar{x}= \frac{1}{n} \sm x_i$ definiert ist):

\begin{align}
b &= \frac{\sm y_i}{n} - \frac{\sm ax_i}{n} \\
	&= \frac{1}{n} \sm y_i - a \frac{1}{n} \sm x_i \\
	&= \bar{y} - a \bar{x}
\end{align}

Setzen wir nun $b=\bar{y} - a \bar{x}$ in Gleichung \ref{eq:parta2} ein, erhalten wir

\begin{equation}
	2\sm x_i \big(ax_i + (\bar{y} - a \bar{x}) - y_i\big)=0
	\label{eq:einsetz}
\end{equation}

\clearpage Durch Ausmultiplizieren und Vereinfachen ergibt sich:

\begin{align}
0 &= 	\sm x_i \big(ax_i + (\bar{y} - a \bar{x}) - y_i\big) \\
	&= \sm \big(ax_i^2+x_i(\bar{y}-a\bar{x})-x_iy_i \big) \\
     &= \sm \big(ax_i^2 + x_i\bar{y} - a\bar{x}x_i- x_iy_i \big) \\
     &= \sm \big(ax_i^2 - a\bar{x}x_i + x_i\bar{y} - x_iy_i \big) 
\end{align}

\begin{align}
     &= \sm \big( (ax_i^2 - a\bar{x}x_i) + x_i\bar{y} - x_iy_i \big) \\
     &= \sm \big( ax_i^2 - a\bar{x}x_i \big) + \sm x_i\bar{y} - \sm x_iy_i 
\end{align}

Jetzt subtrahiert man $\sm x_iy_i$ und addiert $\sm x_i\bar{y}$, um diese beiden Teile auf die andere Seite der Gleichung zu bekommen.

\begin{equation}
\sm (ax_i^2 -ax_i\bar{x}) = \sm x_iy_i - \sm x_i\bar{y} 
\end{equation}

Da $a$ konstant ist, können wir es vor die Klammer ziehen.

\begin{equation}
a \sm (x_i^2 -x_i\bar{x}) = \sm x_iy_i - \bar{y} \sm x_i 
\end{equation}

Jetzt teilen wir durch $\sm (x_i^2 -x_i\bar{x})$

%\begin{equation}
%a \dfrac{\cancel{\sm (x_i^2 -x_i\bar{x})}}{\cancel{\sm (x_i^2 -x_i\bar{x})}}=  \dfrac{\sm x_iy_i - \bar{y} \sm x_i}{\sm (x_i^2 -x_i\bar{x})}
%\end{equation}


\begin{equation}
a =  \dfrac{\sm x_iy_i - \bar{y} \sm x_i}{\sm (x_i^2 -x_i\bar{x})} 
\end{equation}

Aus der Definition des arithmetischen Mittels $\bar{x} = \dfrac{1}{n}\sum x_i$ folgt $\sm x_i= n\bar{x}$. Einsetzen ergibt 

\begin{equation}
a =  \dfrac{\sm x_iy_i - \bar{y} n \bar{x}}{ \sm \big( x_i^2 -  \sm x_i \bar{x} \big)}
\end{equation}


Jetzt zerlegen wir die Summe unter dem Bruchstrich in Einzelsummen und ziehen $\bar{x}$ vor das zweite Summenzeichen (Zur Erinnerung: konstanter Term!)

\begin{equation}
a =  \dfrac{\sm x_iy_i - \bar{y} n\bar{x}}{\sm x_i^2 - \bar{x} \sm x_i}
\end{equation}

Über alternative Formeln zu Varianz und Kovarianz\footnote{Verschiebungssatz: \\ $\cov(X,Y) = \E \left((X - \E(X))(Y - \E(Y)\right)) =\E(X,Y)- \E(X)\E(Y)$ \\ $\var(X)=\E\left(\left(X-\E(X)\right)^2\right)=\E(X^2)-\left(\E(X)\right)^2$} erhalten wir

\begin{equation}
a =  \dfrac{\sm x_iy_i - n\bar{x}\bar{y}}{\sm x_i^2 -n\bar{x}^2} = \dfrac{n \left(\frac{1}{n}\sm x_iy_i - \bar{x}\bar{y}\right)}{ n\left( \frac{1}{n}\sm x_i^2 -\bar{x}^2\right)}  = \dfrac{n\cov(x,y)}{n\var(x)} = \dfrac{\cov(x,y)}{\var(x)}
\end{equation}


\section{Beispiel}

Für unser Beispiel vom Anfang hier die numerische Bestimmung der Parameter. Für $\bar{x}$ erhalten wir $3$, für $\bar{y}=2.4$, die Summe der $(x-\bar{x})(y-\bar{y})$ ergibt $3$, die Summe der $(x-\bar{x})^2=10$. Durch Einsetzen dieser Werte erhalten wir dann als Parameterwert für $b$
1.5, als Parameterwert für $a$ 0.3, sodass die Formel unseres linearen Modells $$y=0.3\cdot x + 1.5$$ lautet.

\begin{center}
\begin{tabular}{r|cccccc} \hline \hline
& 1 & 2 & 3 & 4 & 5 & 6 \\
& $x$	&	$y$	&	$x-\bar{x}$	&	$y-\bar{y}$	&	$(x-\bar{x})(y-\bar{y})$	&	$(x-\bar{x})^2$	\\ \hline
1 & 1	&	1	&	-2	&	-1.4	&	\, 2.8	&	4	\\
2 & 2	&	3	&	-1	&	\, 0.6	&	-0.6	&	1	\\
3 & 3	&	2	&	0	&	-0.4	&	\, 0.0	&	0	\\
4 & 4	&	4	&	1	&	\, 1.6	&	\, 1.6	&	1	\\
5 & 5	&	2	&	2	&	-0.4	&	-0.8	&	4	\\  \hline
$\sum$ & 15 & 12 
\end{tabular}
\end{center}

\section{Quelldateien}

Dieses Dokument wurde mit \LaTeX, dem freien Textsatzsystem, erstellt.

\begin{tabular}{rl}
 \LaTeX & \attachfile{LinearRegressionPrimer.tex} \\
Metapost & \attachfile{linreg_metapost.mp} \\
Metapost (kompiliert) & \attachfile{linreg_metapost.1}
\end{tabular}

\end{document}


\begin{align}
2\sm ax_i + 2\sm b - 2 \sm y_i &= 0 \\
2\sm ax_i \enskip +\enskip  2nb - 2 \sm y_i &= 0 \\
2nb &= 2\sm y_i - 2 \sm ax_i 
\end{align}

\partial

\stackrel{!}{=}0 


















\begin{equation}
	\qs = \sm \Big( \overbrace{y_i^2}^{s^2} - \overbrace{2y_i (ax_i+b)}^{2st} + \overbrace{(ax_i +b)^2}^{t^2}\Big)^2
\end{equation}

Da der Term \((ax_i +b)^2\) der 1. Binomischen Formel entspricht, l�sen wir auch diesen auf:

